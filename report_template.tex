\documentclass{article}
\usepackage[utf8]{inputenc}
\usepackage[russian]{babel}
\usepackage{minted}
\usepackage{graphicx}

\begin{document}
\begin{center}
    \begin{large}
        Лабораторная работа 1 по Теории графов
    \end{large}
\end{center}

% Left: Студент
% Right: Генералов Даниил Михайлович
\begin{tabular}{lr}
    \textbf{Студент:} & \textbf{Генералов Даниил Михайлович} \\
    \textbf{Группа:} & \textbf{НПИ-01-21} \\
    \textbf{Преподаватели:} & \textbf{Доцент Маркова Е.В.,} \\
    \textbf{Тема лабораторной работы:} & \textbf{Алгоритм Флойда-Уоршелла} \\
    \textbf{Количество баллов:} & \textbf{\underline{    } баллов из 10 б.} \\
\end{tabular}

\section{Теоретические сведения по алгоритму Флойда-Уоршелла}

Для лабораторной работы использовались следующие источники:
\begin{itemize}
    \item Алгоритм Флойда — Уоршелла // Википедия. [2022]. Дата обновления: 28.05.2022
    \item Лекции по дискретной математике: Теория графов. Учебное пособие. // Зарипова Э.Р., Кокотчикова М.Г. М., изд-во: РУДН [2013]
\end{itemize}

\section{Реализация алгоритма Флойда-Уоршелла на примере}

Для реализации алгоритма Флойда-Уоршелла, используемого для нахождения кратчайших путей между всеми парами вершин взвешенного графа,
на примере был выбран граф, представленный на рисунке ниже.

\begin{figure}[h]
    \centering
    \includegraphics[width=\textwidth]{graph-init.png}
    \caption{Исходный граф}
\end{figure}

Его матрица расстояний представлена ниже.

\begin{figure}[h]
    \centering
    $$
    \left[
        % <<matrix-init>>
    \right]
    $$
    \caption{Начальная матрица расстояний}
\end{figure}

Для данной матрицы применен представленный ниже исходный код алгоритма Флойда-Уоршелла,
выполненный на языке программирования Python.
\begin{minted}{python}
% <<sourcecode>>
\end{minted}

В результате работы алгоритма получена новая матрица расстояний, представленная на рисунке.
\begin{figure}[h]
    \centering
    $$
    \left[
        % <<matrix-result>>
    \right]
    $$
    \caption{Итоговая матрица расстояний}
\end{figure}

Эту матрицу можно визуализировать как граф, представленный на рисунке.
Вес путей в этом графе равен весу путей в исходном графе.

\begin{figure}[h]
    \centering
    \includegraphics[width=\textwidth]{graph-final.png}
    \caption{Граф, полученный в результате работы алгоритма}
\end{figure}

\end{document}